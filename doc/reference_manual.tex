\documentclass[12pt]{article}


\newcommand{\funcname}{\begingroup \catcode`_=12 \funcnameIn}
\newcommand{\funcnameIn}[2]{
\begin{tabular}{l}
{\Large\bf \texttt{#1}\label{#1}}\\
{\small\it #2}\\ 
\end{tabular}
}
\newcommand{\location}[1]{
\begin{tabular}{l}
{\large at} \texttt{#1}
\end{tabular}
}
\renewcommand{\description}[1]{
Description: #1
}
\newcommand{\inputargs}[2]{
$\rightarrow$
{\bf \texttt{#1}}: {#2} 
}
\newcommand{\outputargs}[2]{
$\leftarrow$
{\bf \texttt{#1}}: {#2}
}
\newcommand{\example}[1]{
Eg.: \texttt{#1}
}
\newcommand{\seealso}[1]{
See also: {\bf \texttt{#1}} - \ref{#1}
}



\begin{document}

% function
\funcname{create_rtp_l1b}{matlab function}

\location{git/prod_mat/airs/rtp}

\description{Create an AIRS allfovs files, on a granule base}

\inputargs{date}{a particular data in matlab format (range)}

\inputargs{granule}{(optional) the desired granule number array (empty means all)}

\inputargs{overwrite}{(optional) 0-don't, 1-do.}

\inputargs{output_dir}{(optional) overide the default dump directory.}

\outputargs{}



\example{create_rtp_l1b(datenum(2012,09,20)); \% all grans}

\example{create_rtp_l1b(datenum(2012,09,20:22),[]);  \% two days, all grans}

\example{create_rtp_l1b(datenum(2012,09,20),[16:24]); \% grans 16-24}

\seealso{create_rtp_l1b}





\end{document}
